\documentclass[12pt]{article}
\usepackage{lingmacros}
\usepackage{tree-dvips}
\usepackage{graphicx} %package to manage images
\usepackage{float}
\usepackage{subcaption}%package for enabling multiple images in a single figure
\graphicspath{ {/} }
\begin{document}

\section*{Question 2}

\subsection*{Part a}

1) 4.0 GPA - P, the data match the decision tree\\
2) 3.9 GPA - P, the data match the decision tree\\
3) 3.9 GPA - P, the data match the decision tree\\
4) 3.8 GPA - yes publications - P, the data match the decision tree\\
5) 3.6 GPA - no publications - rank 2 university - P, the data match the decision tree\\
6) 3.6 GPA - yes publications - P, the data match the decision tree\\
7) 3.4 GPA - no publications - rank 3 university - N, the data match the decision tree\\
8) GPA 3.4 - No publication - Rank 1 University - N, data match the tree\\
9) GPA 3.2 - N, data match the tree\\
10) GPA 3.1 - N, data match the tree\\
11) GPA 3.1 - N, data match the tree\\
12) GPA 3.0 - N, data match the tree\\

\subsection*{Part b}

For GPA, the information gained is:

\begin{equation}
I(4.0, 3.6, 3.3) = -\frac{1}{4}\log_3\frac{1}{4} -\frac{5}{12}\log_3\frac{5}{12} -\frac{1}{3}\log_3\frac{1}{3}
\end{equation}
\begin{equation}
I(4.0, 3.6, 3.3) = 0.3155 + 0.3320 + 0.3333 = 0.9808
\end{equation}

For university rank, the information gained is:

\begin{equation}
I(rank 1, rank 2, rank 3) = -\frac{5}{12}\log_3\frac{5}{12} -\frac{1}{4}\log_3\frac{1}{4} -\frac{1}{3}\log_3\frac{1}{3}
\end{equation}
\begin{equation}
I(rank 1, rank 2, rank 3) = 0.3320 + 0.3155 + 0.3333 = 0.9808
\end{equation}


\end{document}